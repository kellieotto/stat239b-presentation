\documentclass{beamer}

 \usepackage{beamerthemesplit}
 
 
% Math macros
\newcommand{\cD}{{\mathcal D}}
\newcommand{\cF}{{\mathcal F}}
\newcommand{\todo}[1]{{\color{red}{TO DO: \sc #1}}}

\newcommand{\reals}{\mathbb{R}}
\newcommand{\integers}{\mathbb{Z}}
\newcommand{\naturals}{\mathbb{N}}
\newcommand{\rationals}{\mathbb{Q}}

\newcommand{\ind}[1]{1{\{#1\}}} % Indicator function
\newcommand{\pr}{\mathbb{P}} % Generic probability
\newcommand{\ex}{\mathbb{E}} % Generic expectation
\newcommand{\var}{\textrm{Var}}
\newcommand{\cov}{\textrm{Cov}}

\newcommand{\normal}{N} % for normal distribution (can probably skip this)
\newcommand{\eps}{\varepsilon}
\newcommand\independent{\protect\mathpalette{\protect\independenT}{\perp}}
\def\independenT#1#2{\mathrel{\rlap{$#1#2$}\mkern2mu{#1#2}}}

\newcommand{\convd}{\stackrel{d}{\longrightarrow}} % convergence in distribution/law/measure
\newcommand{\convp}{\stackrel{P}{\longrightarrow}} % convergence in probability
\newcommand{\convas}{\stackrel{\textrm{a.s.}}{\longrightarrow}} % convergence almost surely

\newcommand{\eqd}{\stackrel{d}{=}} % equal in distribution/law/measure
\newcommand{\argmax}{\arg\!\max}
\newcommand{\argmin}{\arg\!\min}
 \newcommand{\bit}{\begin{itemize}}
 \newcommand{\eit}{\end{itemize}}
 
 
%%%%%%%%%%%%%%%%%%%%%%%%%%%%%%%%%%%%%%%%%%%%%

\title{On the Failure of the Bootstrap for Matching Estimators}
\author{Andrew Do, Kellie Ottoboni, Simon Walter}
\date{April 8, 2016}

\begin{document}

\frame{\titlepage}

\section[Outline]{}
\frame{\tableofcontents}

\section{Introduction}

\frame{
\frametitle{Problem statement}
\todo{problem of finding standard errors for an estimator

asymptotic results are sometimes available}
}

\frame
{
  \frametitle{The bootstrap}
\todo{nifty slide or two explaining the bootstrap. Mike Jordan had a nice graphic in his bag of little bootstraps talk.}
}




\section{Abadie and Imbens (2008)}

\frame{
\frametitle{On the failure of the bootstrap}
Thesis: \todo

}

\subsection{Notation and Assumptions}

\frame{
\frametitle{Notation and Assumptions}
\begin{itemize}
\item Suppose we have a random sample of $N_0$ units from the control population and a random sample of $N_1$ units from the treated population, with $N = N_0 + N_1$
\item Each unit has a pair of potential outcomes, $Y_i(0)$ and $Y_i(1)$, under the control and active treatments
\item Let $W_i$ indicate treatment: we observe $Y_i = W_i Y_i(1) + (1-W_i) Y_i(0)$
\item In addition to the outcome, we observe a (scalar) covariate $X_i$ for each individual
\end{itemize}
We're interested in the \textbf{average treatment effect for the treated} (ATT):

$$\tau = \ex(Y_i(1) - Y_i(0) \mid W_i = 1)$$
}


\frame{
\frametitle{Notation and Assumptions}

We make the usual assumptions for matching:

\begin{itemize}
\item Unconfoundedness: For almost all $x$,
$$(Y_i(0), Y_i(1)) \independent W_i \mid X_i = x \text{almost surely}$$
\item Overlap: For some $c>0$ and almost all $x$,
$$c \leq \pr(W_i = 1 \mid X_i = x) \leq 1-c$$
\end{itemize}
}



\frame{
\frametitle{Notation and Assumptions}
$D_i$ is the distance between the covariate values for observation $i$ and the closest control group match:

$$D_i = \min_{j = 1, \dots, N: W_j = 0} \left\Vert X_i - X_j \right\Vert$$

$\mathcal{J}(i)$ is the set of closest matches for treated unit $i$. 

\begin{displaymath}
   \mathcal{J}(i) = \left\{
     \begin{array}{lr}
       \{ j \in \{1, \dots, N\} : W_j = 0, \left\Vert X_i - X_j \right\Vert = D_i \} & \text{ if  } W_i = 1\\
       \emptyset & \text{ if  } W_i = 0
     \end{array}
   \right.
\end{displaymath} 

If $X$ is continuous, this set will consist of one unit with probability 1. In bootstrap samples, units may appear more than once.
}


\frame{
\frametitle{Notation and Assumptions}
Estimate the counterfactual for each treated unit as:

$$\hat{Y}_i(0) = \frac{1}{\# \mathcal{J}(i)} \sum_{j \in \mathcal{J}(i)} Y_i$$

The matching estimator of $\tau$ is then

$$\hat{\tau} = \frac{1}{N_1} \sum_{i : W_i = 1} \left(Y_i - \hat{Y}_i(0)\right)$$
}




\frame{
\frametitle{Notation and Assumptions}
An alternative way of writing the estimator is

$$\hat{\tau} = \frac{1}{N_1} \sum_{i=1}^N (W_i - (1-W_i)K_i) Y_i$$

where $K_i$ is the weighted number of times that unit $i$ is used as a match:

\begin{displaymath}
   K_i = \left\{
     \begin{array}{lr}
      0 & \text{ if  } W_i = 1\\
      \sum_{j: W_j=1} \ind{i \in \mathcal{J}(j)} \frac{1}{\#\mathcal{J}(j)} & \text{ if  } W_i = 0
     \end{array}
   \right.
\end{displaymath} 
}



\subsection{The Bootstrap}

\frame{
\frametitle{Bootstrap}
\begin{itemize}
\item Think of $Z = (X, W, Y)$ as a random sample and $t(\cdot)$ as a functional on $Z$. $\hat{\tau} = t(Z)$. 
\item We obtain a \textbf{bootstrap sample} $Z_b$ by taking a random sample with replacement from $Z$. 
\item We calculate the bootstrap estimator by applying $t(\cdot)$ to $Z_b$: $\hat{\tau}_b = t(Z_b)$.
\end{itemize}
}


\frame{
\frametitle{Bootstrap}
The bootstrap variance of $\hat{\tau}$ is the variance of $\hat{\tau}_b$ conditional on $Z$:

$$V^{B} = \ex\left[ (\hat{\tau}_b - \hat{\tau})^2 \mid Z\right]$$

We estimate it by generating $B$ bootstrap samples from $Z$ and taking the following average:

$$\hat{V}^B = \frac{1}{B} \sum_{b=1}^B \left( \hat{\tau}_b - \hat{\tau} \right)^2$$
}

\frame{
\frametitle{Bootstrap}
\textbf{Issue:} the bootstrap fails to replicate the distribution of $K_i$, even in large samples
\begin{itemize}
\item Suppose the ratio $N_1/N_0$ is small (i.e. there are few treated relative to controls)
\item In the original sample, few controls are used as a match more than once
\item In bootstrap samples, treated units may appear multiple times, creating situations where $\pr(K_{b,i} > 1) > \pr(K_i > 1)$ \todo{is this technically correct? is there a better way to put this?}
\end{itemize}
}


\frame{
\frametitle{Bootstrap}
some theory?
}

\frame{
\frametitle{Prior work on the inconsistency of the bootstrap}
\bit
\item Abadie and Imbens claim  this is the first case for which the bootstrap is inconsistent for a statistic that is asymptotically normal and root-$n$ consistent.
\item However, there is a prior example on this point. Beran (1982) establishes that a Hodges-type estimator for the  mean:

$$\theta(X_1, \ldots, X_n) = \begin{cases} 
b \bar{X}_n \textrm{  if $ |\bar{X}_n| < n^{-1/4}$} \\
\bar{X}_n \textrm{  if $ |\bar{X}_n| \geq n^{-1/4}$}
\end{cases}$$
is not consistently estimated by the bootstrap when the true mean is zero.
\todo{This mistake only exists in the preprint not the final version :(}
\eit
}

\frame{
\bit
\item There is a widespread but imperfect intuition that states that Efron's bootstrap consistently estimates the distribution of the statistic  if and only if the statistic is asymptotically normal. 
\item Has a rigorous proof for statistics that are linear in the data but the general case defied rigorous proof.
\item  Rigorous results for non-linear statistics seem to require that small changes in the population functional lead to small changes in the asymptotic distribution of the statistic. 
\item For example in the case of previous slide it is not hard to show this is violated:
\begin{align*}
\theta(X_1, \ldots, X_n) \Longrightarrow
N\left[\mu, \frac{1}{n} \var(X))\right] \textrm{\quad   if $E(X) \neq 0$} \\
\theta(X_1, \ldots, X_n) \Longrightarrow N\left[\mu, \frac{b^2}{n} \var(X)\right] \textrm{\quad   if $E(X) = 0$}
\end{align*}

\eit
}




\frame{
\frametitle{Should we follow Abadie and Imbens recommendation?}
\bit
\item Conclusion is that only their prior work based on asymptotic normality and a subsampling approach have formal justification and may therefore be preferred.
\todo{This claim is also weakened in the final version; I really should have read that one!}
\item This is not satisfying. The bootstrap is used because it is second order correct:
\todo{describe what second order correctness is}
\item \todo{Suggest that multiplier bootstrap ?is?  second order correct and therefore might be preferred. }
\eit

}


\section{Simulations}





\end{document}
